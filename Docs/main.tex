%%%%%%%%%%%%%%%%%%%%%%%%%%%%%%%%%%%%%%%%%
% Lachaise Assignment
% LaTeX Template
% Version 1.0 (26/6/2018)
%
% This template originates from:
% http://www.LaTeXTemplates.com
%
% Authors:
% Marion Lachaise & François Févotte
% Vel (vel@LaTeXTemplates.com)
%
% License:
% CC BY-NC-SA 3.0 (http://creativecommons.org/licenses/by-nc-sa/3.0/)
% 
%%%%%%%%%%%%%%%%%%%%%%%%%%%%%%%%%%%%%%%%%

%----------------------------------------------------------------------------------------
%	PACKAGES AND OTHER DOCUMENT CONFIGURATIONS
%----------------------------------------------------------------------------------------

\documentclass{article}
\usepackage{hyperref}
\usepackage{array}
%%%%%%%%%%%%%%%%%%%%%%%%%%%%%%%%%%%%%%%%%
% Lachaise Assignment
% Structure Specification File
% Version 1.0 (26/6/2018)
%
% This template originates from:
% http://www.LaTeXTemplates.com
%
% Authors:
% Marion Lachaise & François Févotte
% Vel (vel@LaTeXTemplates.com)
%
% License:
% CC BY-NC-SA 3.0 (http://creativecommons.org/licenses/by-nc-sa/3.0/)
% 
%%%%%%%%%%%%%%%%%%%%%%%%%%%%%%%%%%%%%%%%%

%----------------------------------------------------------------------------------------
%	PACKAGES AND OTHER DOCUMENT CONFIGURATIONS
%----------------------------------------------------------------------------------------

\usepackage{amsmath,amsfonts,stmaryrd,amssymb} % Math packages

\usepackage{enumerate} % Custom item numbers for enumerations

\usepackage[ruled]{algorithm2e} % Algorithms

\usepackage[framemethod=tikz]{mdframed} % Allows defining custom boxed/framed environments

\usepackage{listings} % File listings, with syntax highlighting
\lstset{
	basicstyle=\ttfamily, % Typeset listings in monospace font
}

%----------------------------------------------------------------------------------------
%	DOCUMENT MARGINS
%----------------------------------------------------------------------------------------

\usepackage{geometry} % Required for adjusting page dimensions and margins

\geometry{
	paper=a4paper, % Paper size, change to letterpaper for US letter size
	top=2.5cm, % Top margin
	bottom=3cm, % Bottom margin
	left=2.5cm, % Left margin
	right=2.5cm, % Right margin
	headheight=14pt, % Header height
	footskip=1.5cm, % Space from the bottom margin to the baseline of the footer
	headsep=1.2cm, % Space from the top margin to the baseline of the header
	%showframe, % Uncomment to show how the type block is set on the page
}

%----------------------------------------------------------------------------------------
%	FONTS
%----------------------------------------------------------------------------------------

\usepackage[utf8]{inputenc} % Required for inputting international characters
\usepackage[T1]{fontenc} % Output font encoding for international characters

\usepackage{XCharter} % Use the XCharter fonts

%----------------------------------------------------------------------------------------
%	COMMAND LINE ENVIRONMENT
%----------------------------------------------------------------------------------------

% Usage:
% \begin{commandline}
%	\begin{verbatim}
%		$ ls
%		
%		Applications	Desktop	...
%	\end{verbatim}
% \end{commandline}

\mdfdefinestyle{commandline}{
	leftmargin=10pt,
	rightmargin=10pt,
	innerleftmargin=15pt,
	middlelinecolor=black!50!white,
	middlelinewidth=2pt,
	frametitlerule=false,
	backgroundcolor=black!5!white,
	frametitle={Command Line},
	frametitlefont={\normalfont\sffamily\color{white}\hspace{-1em}},
	frametitlebackgroundcolor=black!50!white,
	nobreak,
}

% Define a custom environment for command-line snapshots
\newenvironment{commandline}{
	\medskip
	\begin{mdframed}[style=commandline]
}{
	\end{mdframed}
	\medskip
}

%----------------------------------------------------------------------------------------
%	FILE CONTENTS ENVIRONMENT
%----------------------------------------------------------------------------------------

% Usage:
% \begin{file}[optional filename, defaults to "File"]
%	File contents, for example, with a listings environment
% \end{file}

\mdfdefinestyle{file}{
	innertopmargin=1.6\baselineskip,
	innerbottommargin=0.8\baselineskip,
	topline=false, bottomline=false,
	leftline=false, rightline=false,
	leftmargin=2cm,
	rightmargin=2cm,
	singleextra={%
		\draw[fill=black!10!white](P)++(0,-1.2em)rectangle(P-|O);
		\node[anchor=north west]
		at(P-|O){\ttfamily\mdfilename};
		%
		\def\l{3em}
		\draw(O-|P)++(-\l,0)--++(\l,\l)--(P)--(P-|O)--(O)--cycle;
		\draw(O-|P)++(-\l,0)--++(0,\l)--++(\l,0);
	},
	nobreak,
}

% Define a custom environment for file contents
\newenvironment{file}[1][File]{ % Set the default filename to "File"
	\medskip
	\newcommand{\mdfilename}{#1}
	\begin{mdframed}[style=file]
}{
	\end{mdframed}
	\medskip
}

%----------------------------------------------------------------------------------------
%	NUMBERED QUESTIONS ENVIRONMENT
%----------------------------------------------------------------------------------------

% Usage:
% \begin{question}[optional title]
%	Question contents
% \end{question}

\mdfdefinestyle{question}{
	innertopmargin=1.2\baselineskip,
	innerbottommargin=0.8\baselineskip,
	roundcorner=5pt,
	nobreak,
	singleextra={%
		\draw(P-|O)node[xshift=1em,anchor=west,fill=white,draw,rounded corners=5pt]{%
		Question \theQuestion\questionTitle};
	},
}

\newcounter{Question} % Stores the current question number that gets iterated with each new question

% Define a custom environment for numbered questions
\newenvironment{question}[1][\unskip]{
	\bigskip
	\stepcounter{Question}
	\newcommand{\questionTitle}{~#1}
	\begin{mdframed}[style=question]
}{
	\end{mdframed}
	\medskip
}

%----------------------------------------------------------------------------------------
%	WARNING TEXT ENVIRONMENT
%----------------------------------------------------------------------------------------

% Usage:
% \begin{warn}[optional title, defaults to "Warning:"]
%	Contents
% \end{warn}

\mdfdefinestyle{warning}{
	topline=false, bottomline=false,
	leftline=false, rightline=false,
	nobreak,
	singleextra={%
		\draw(P-|O)++(-0.5em,0)node(tmp1){};
		\draw(P-|O)++(0.5em,0)node(tmp2){};
		\fill[black,rotate around={45:(P-|O)}](tmp1)rectangle(tmp2);
		\node at(P-|O){\color{white}\scriptsize\bf !};
		\draw[very thick](P-|O)++(0,-1em)--(O);%--(O-|P);
	}
}

% Define a custom environment for warning text
\newenvironment{warn}[1][Warning:]{ % Set the default warning to "Warning:"
	\medskip
	\begin{mdframed}[style=warning]
		\noindent{\textbf{#1}}
}{
	\end{mdframed}
}

%----------------------------------------------------------------------------------------
%	INFORMATION ENVIRONMENT
%----------------------------------------------------------------------------------------

% Usage:
% \begin{info}[optional title, defaults to "Info:"]
% 	contents
% 	\end{info}

\mdfdefinestyle{info}{%
	topline=false, bottomline=false,
	leftline=false, rightline=false,
	nobreak,
	singleextra={%
		\fill[black](P-|O)circle[radius=0.4em];
		\node at(P-|O){\color{white}\scriptsize\bf i};
		\draw[very thick](P-|O)++(0,-0.8em)--(O);%--(O-|P);
	}
}

% Define a custom environment for information
\newenvironment{info}[1][Info:]{ % Set the default title to "Info:"
	\medskip
	\begin{mdframed}[style=info]
		\noindent{\textbf{#1}}
}{
	\end{mdframed}
}
 % Include the file specifying the document structure and custom commands

%----------------------------------------------------------------------------------------
%	ASSIGNMENT INFORMATION
%----------------------------------------------------------------------------------------

\title{MPRI Graph Mining: Find Quasi Cliques}

\author{Mohammad Mahzoun\\ \texttt{Mohammad.Mahzoun@gmail.com}} % Author name and email address

\date{\today} % University, school and/or department name(s) and a date

%----------------------------------------------------------------------------------------

\begin{document}

\maketitle % Print the title

%----------------------------------------------------------------------------------------
%	INTRODUCTION
%----------------------------------------------------------------------------------------

\section{Introduction} % Unnumbered section
Our aim in this project is to find $\alpha-quasi-cliques$ with at least $s$ nodes using a given heuristic approach. In this document, the running time of the implementation for different graphs is reported. 

\subsection{Hardware}
The system we are using to run the code is running a Ubuntu 18.04 operating system with 16GB of memory and Intel COREI7 8th Gen CPU. 


\subsection{Implementation} % Numbered section
The code is publicly available on \href{https://github.com/mahzoun/MPRIGraphMining}{Github}.

%------------------------------------------------

\section{Results}
In the following, we use different graphs as the input to algorithm. The values $k \in \{2, 3, 4\}$ and $s = 10$ are used for the required parameters. In each section, the results for each graph is described. 

	
%------------------------------------------------

\subsection{\href{http://konect.uni-koblenz.de/networks/arenas-pgp}{Pretty Good Privacy}}
This is the interaction network of users of the Pretty Good Privacy (PGP) algorithm. The network contains only the giant connected component of the network. The graph is undirected with $10680$ nodes and $24316$ edges. 
\begin{center}
	\begin{tabular}{ | m{1cm} | m{1cm}| m{2cm} | m{1cm} | } 
		\hline
		$k$ & $result$ & $time$ & $size$\\ 
		\hline
		$2$ & $1$ & $95s$  & $24$\\ 
		\hline
		$3$ & $1$ & $98s$  & $24$\\	 
		\hline
		$4$ & $1$ & $100s$  & $24$\\
		\hline
	\end{tabular}
\end{center}
By increasing the value of $k$, the running time increase because the graph is sparse and calculating the $k-degree$ is slower by the increasing of $k$. The result is same because there is a complete sub-graph of order $24$.

\subsection{\href{http://konect.uni-koblenz.de/networks/ca-AstroPh}{Route views}}
This is the undirected network of autonomous systems of the Internet connected with each other. Nodes are autonomous systems (AS), and edges denote communitation. The network contains loops. The graph is undirected with $6474$ nodes and $13895$ edges. 
\begin{center}
	\begin{tabular}{ | m{1cm} | m{1cm}| m{2cm} | m{1cm} | } 
		\hline
		$k$ & $result$ & $time$ & $size$\\ 
		\hline
		$2$ & $1$ & $25s$  & $10$\\ 
		\hline
		$3$ & $1$ & $110s$  & $10$\\	 
		\hline
		$4$ & $1$ & $250s$  & $10$\\
		\hline
	\end{tabular}
\end{center}
By increasing the value of $k$, the running time increase because the graph is sparse and calculating the $k-degree$ is slower by the increasing of $k$. The result is same because there is a complete sub-graph of order $24$.


\subsection{\href{http://konect.uni-koblenz.de/networks/opsahl-powergrid}{US power grid}}
This undirected network contains information about the power grid of the Western States of the United States of America. An edge represents a power supply line. A node is either a generator, a transformator or a substation. The network contains loops. The graph is undirected with $4941$ nodes and $6594$ edges. 
\begin{center}
	\begin{tabular}{ | m{1cm} | m{1cm}| m{2cm} | m{1cm} | } 
		\hline
		$k$ & $result$ & $time$ & $size$\\ 
		\hline
		$2$ & $0.6$ & $10s$  & $10$\\ 
		\hline
		$3$ & $0.57$ & $20s$  & $10$\\	 
		\hline
		$4$ & $0.55$ & $28s$  & $10$\\
		\hline
	\end{tabular}
\end{center}
By increasing the value of $k$, the running time increase because the graph is sparse and calculating the $k-degree$ is slower by the increasing of $k$. The result is different because the order of nodes in deleting has been changed.


\section{Improvement}
The bottleneck of the algorithm is listing all $k-cliques$. Minor implementation tunings will result in better performance. However, for small values of $k$, we may do better. For $k == 2$, we may just use the degree of each vertex which is faster. 
\end{document}
